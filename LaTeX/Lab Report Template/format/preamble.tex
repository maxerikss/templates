\usepackage[utf8]{inputenc}


% Set your default font encoding.
\usepackage[T1]{fontenc}

% Set default language priority order.
\usepackage[english]{babel}

% Set default margins.
\usepackage[a4paper,margin=25mm,headsep=5mm,headheight=12pt]{geometry}

\usepackage{amsmath,amssymb,amsfonts,amsthm}  % Mathematics
\usepackage{bbold}

\usepackage{hyperref} 
\hypersetup{colorlinks, 
    citecolor=black,
    filecolor=black, 
    linkcolor=black,
    urlcolor=black
      }

\usepackage{setspace}
\usepackage{graphicx}
\usepackage{subcaption}
\usepackage[separate-uncertainty=true]{siunitx} %Use this for writing SI units
\sisetup{
  range-phrase=--,
  detect-all,
  output-decimal-marker={.},
  range-units=single,
  per-mode=reciprocal,
  separate-uncertainty=false
}
\usepackage{physics} %A lot of useful short commands for writing math/physics
\usepackage{url}
\usepackage{verbatim}
\usepackage[toc,acronym]{glossaries}
\usepackage{booktabs} % For nice tables
\usepackage{fancyhdr} % For header and footer
\usepackage{lipsum}
\usepackage{lastpage} % For page counter
\usepackage{svg} % Makes it possible to use .svg files 
\usepackage{float} % Just [H] for placing figures
\usepackage{comment} % Use \begin{comment}
\usepackage{listings} % Enables source code listings
\usepackage{enumerate}
\usepackage{enumitem}
\definecolor{codegreen}{rgb}{0,0.75,0}
\definecolor{codegray}{rgb}{0.5,0.5,0.5}
\definecolor{codepurple}{rgb}{0.58,0,0.82}
\definecolor{backcolour}{rgb}{0.97,0.96,0.97}
\lstdefinestyle{mystyle}{
    backgroundcolor=\color{backcolour},   
    commentstyle=\color{codegreen},
    keywordstyle=\color{codepurple},
    numberstyle=\tiny\color{codegray},
    stringstyle=\color{magenta},
    basicstyle=\ttfamily\footnotesize,
    breakatwhitespace=false,         
    breaklines=true,                 
    captionpos=b,                    
    keepspaces=true,                 
    numbers=left,                    
    numbersep=5pt,                  
    showspaces=false,                
    showstringspaces=false,
    showtabs=false,                  
    tabsize=2
}

\lstset{style=mystyle}

% using inkscapefiles
\graphicspath{{figures/}}
